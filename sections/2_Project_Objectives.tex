\section{Project Objectives \& Scope}
The main objective of this project is to gain hands on experience implementing and analyzing different scheduling algorithms for different loads. Specifically, we will analyze performance of (m,k)-firm RMS, (m,k)-firm EDF, and (m,k)-firm DBP for different overload situations. We compare them against base EDF in terms of preemption rate, deadline miss ratio, and fraction of windows that satisfy (m,k) requirements. By studying the performance of these algorithms in overloaded scenarios, we can better understand the tradeoffs of the algorithms.

\subsection{Problem Statement and System Model}
In real world real-time applications, some task allow for missed jobs, as long as the misses are adequately spaced. As mentioned in the introduction, and \cite{mkdbp-og}, the anti-lock braking system is one example of such system. The (m,k)-firm notion is therefore used to describe such system. System with tasks with (m,k)-firm requirements can show up in a wide range of applications, from embedded systems, to communication. Therefore, in this project, we will implement and analyze different algorithms for (m,k)-firm tasks and discuss the tradeoffs of them for different applications.

\subsection{Objectives and Scope}
The main objective is to implement (m,k)-firm RMS, EDF, and DBP algorithms and perform analysis on each of them. We will then compare them, using metrics like preemption rate, deadline miss ratio, and fraction of windows that satisfy (m,k) requirements, against a baseline algorithm, EDF. And, finally, we will discuss the tradeoffs between them for different applications.
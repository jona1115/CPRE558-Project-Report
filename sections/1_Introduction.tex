\section{Introduction}
Classical RMS and EDF schedulers are designed under the assumption that all jobs of all periodic tasks must always meet their deadlines. In practice, there are many tasks where this assumption is unrealistic and that tasks might temporarily or permanently overload the CPU.

However, there exist many real-time applications where meeting all task deadlines are not necessary as long as the misses are adequately spaced. For example, as mentioned in \cite{mkdbp-og}, in an anti-lock braking system, a real-time task typically determines the onset of locking by repeatedly sampling the rotational speed of each wheel. However, since the speed of a wheel can be projected from a recent history of speeds, it is usually not necessary for every instance of this task to meet its deadline as long as not too many consecutive instance of the task miss their deadlines. Because of this, in \cite{mkidea}, Hamsaoui and Ramanathan introduce the notion of (m,k)-firm deadline to better characterize timing constrains of real-time tasks where requirement is not violated as long as m out of k consecutive jobs are not missed.

In this project, the goal is to implement (m,k)-firm RMS, EDF, and DBP algorithms and perform analysis on each of them. (m,k)-firm DBP is implemented as described by Goossens in \cite{goossens:inria-00336461}, which is a slight modification of the original (m,k)-firm DBP algorithm described by Hamsaoui and Ramanathan in \cite{mkdbp-og}.
